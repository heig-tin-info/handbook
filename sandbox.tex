\documentclass[a4paper]{article}
\usepackage{xcolor}
\usepackage{tikz}
\usepackage{amsmath}
\usepackage{stackengine}
\usepackage{hyperref}
\usepackage{ulem}
\usepackage{listings}
\usepackage{lipsum}
\usepackage{graphicx}
\usepackage{fontspec}
\usepackage{unicode-math} % For using Unicode math symbols
\usepackage[most]{tcolorbox}
\usepackage{fontawesome5}
\usepackage{parskip}
\usepackage{stackengine}
\usepackage{minted}

\tcbuselibrary{listingsutf8}

\newtcblisting{pythoncode}[2][]{colback=gray!5!white, colframe=gray!75!black,
    listing only, listing options={style=python, breaklines=true, basicstyle=\ttfamily, numbers=left, numberstyle=\tiny}, title=#2, #1}


\newtcolorbox{callout}[2][]{
    breakable,
    sharp corners,
    skin=enhancedmiddle
    jigsaw,
    parbox=false,
    boxrule=0mm,
    leftrule=2mm,
    boxsep=0mm,
    arc=0mm,
    outer arc=0mm,
    attach title to upper,
    after title={.\ },
    coltitle=black,
    colback=gray!10,
    colframe=black,
    title={\faLaptopCode\quad #2},
    fonttitle=\bfseries,
    #1
}

\usetikzlibrary{shadows}

\newcommand*\kbd[1]{%
  \tikz[baseline=(key.base)]
    \node[%
      draw,
      fill=white,
      drop shadow={shadow xshift=0.25ex,shadow yshift=-0.25ex,fill=black,opacity=0.75},
      rectangle,
      rounded corners=2pt,
      inner sep=1pt,
      line width=0.5pt,
      font=\scriptsize\sffamily
    ](key) {#1\strut}
  ;
}

\makeatletter

% here is a macro expanding to the name of the language
% (handy if you decide to change it further down the road)
\newcommand\language@yaml{yaml}

\expandafter\expandafter\expandafter\lstdefinelanguage
\expandafter{\language@yaml}
{
  keywords={true,false,null,y,n},
  keywordstyle=\color{darkgray}\bfseries,
  basicstyle=\YAMLkeystyle,                                 % assuming a key comes first
  sensitive=false,
  comment=[l]{\#},
  morecomment=[s]{/*}{*/},
  commentstyle=\color{purple}\ttfamily,
  stringstyle=\YAMLvaluestyle\ttfamily,
  moredelim=[l][\color{orange}]{\&},
  moredelim=[l][\color{magenta}]{*},
  moredelim=**[il][\YAMLcolonstyle{:}\YAMLvaluestyle]{:},   % switch to value style at :
  morestring=[b]',
  morestring=[b]",
  literate =    {---}{{\ProcessThreeDashes}}3
                {>}{{\textcolor{red}\textgreater}}1
                {|}{{\textcolor{red}\textbar}}1
                {\ -\ }{{\mdseries\ -\ }}3,
}

% switch to key style at EOL
\lst@AddToHook{EveryLine}{\ifx\lst@language\language@yaml\YAMLkeystyle\fi}
\makeatother

\newcommand\correctchoice{\stackinset{c}{}{c}{}{$\times$}{$\bigcirc$}}
\newcommand\choice{$\bigcirc$}


% Définir une nouvelle commande pour surligner le texte avec un encadré
\newtcbox{\hl}{on line, boxrule=0.5pt, colframe=black, colback=yellow!30, sharp corners, boxsep=0pt, left=2pt, right=2pt, top=2pt, bottom=2pt}

\begin{document}

\lipsum[2]


\section{Example}
Lorem ipsum dolor sit\textsuperscript{amet}, consectetur adipiscing elit. \sout{Sed non risus}. Suspendisse
  \emph{lectus} tortor, \textbf{dignissim} sit amet, adipiscing nec, ultricies sed, dolor. H\textsubscript{2}O.
  Cras elementum 36 cm\textsuperscript{3}. \uline{ ultrices } diam. Maecenas ligula massa, varius a, semper
  congue, euismod non, mi. Proin porttitor, orci nec nonummy molestie, enim est \hl{eleifend mi},
  non fermentum diam nisl sit amet erat. Duis semper. Duis arcu massa, scelerisque vitae, consequat in, pretium a, enim.
  \texttt{Pellentesque congue}. Ut in risus volutpat libero pharetra tempor. Cras vestibulum bibendum augue.
  Praesent egestas leo in pede. Praesent blandit odio eu enim. Pellentesque sed dui ut augue blandit sodales. Vestibulum
  ante ipsum primis in faucibus orci luctus et ultrices posuere cubilia Curae ; Aliquam nibh. Mauris ac \lstinline[language=c]{mauris sed}
  pede pellentesque fermentum. Maecenas adipiscing ante non diam sodales hendrerit.

% [^1]: This is a footnote.

\subsection{Lists}
\subsubsection{Unordered list}
\begin{itemize}
\item
Nulla et rhoncus turpis. Mauris ultricies elementum leo. Duis efficitur
      accumsan nibh eu mattis. Vivamus tempus velit eros, porttitor placerat nibh
      lacinia sed. Aenean in finibus diam.


Duis mollis est eget nibh volutpat, fermentum aliquet dui mollis.
Nam vulputate tincidunt fringilla.
Nullam dignissim ultrices urna non auctor.


\item Duis mollis est eget nibh volutpat, fermentum aliquet dui mollis.
\item Nam vulputate tincidunt fringilla.
\item Nullam dignissim ultrices urna non auctor.
\end{itemize}

\subsubsection{Ordered list}
\begin{enumerate}
\item
Vivamus id mi enim. Integer id turpis sapien. Ut condimentum lobortis
      sagittis. Aliquam purus tellus, faucibus eget urna at, iaculis venenatis
      nulla. Vivamus a pharetra leo.



Vivamus venenatis porttitor tortor sit amet rutrum. Pellentesque aliquet
          quam enim, eu volutpat urna rutrum a. Nam vehicula nunc mauris, a
          ultricies libero efficitur sed.



Morbi eget dapibus felis. Vivamus venenatis porttitor tortor sit amet
          rutrum. Pellentesque aliquet quam enim, eu volutpat urna rutrum a.


Mauris dictum mi lacus
Ut sit amet placerat ante
Suspendisse ac eros arcu




\item
Vivamus venenatis porttitor tortor sit amet rutrum. Pellentesque aliquet
          quam enim, eu volutpat urna rutrum a. Nam vehicula nunc mauris, a
          ultricies libero efficitur sed.


\item
Morbi eget dapibus felis. Vivamus venenatis porttitor tortor sit amet
          rutrum. Pellentesque aliquet quam enim, eu volutpat urna rutrum a.


Mauris dictum mi lacus
Ut sit amet placerat ante
Suspendisse ac eros arcu


\item Mauris dictum mi lacus
\item Ut sit amet placerat ante
\item Suspendisse ac eros arcu
\end{enumerate}

\subsubsection{Definition list}
\begin{description}

\item[\texttt{Lorem ipsum dolor sit amet}]

Sed sagittis eleifend rutrum. Donec vitae suscipit est. Nullam tempus
      tellus non sem sollicitudin, quis rutrum leo facilisis.



\item[\texttt{Cras arcu libero}]

Aliquam metus eros, pretium sed nulla venenatis, faucibus auctor ex. Proin
      ut eros sed sapien ullamcorper consequat. Nunc ligula ante.

Duis mollis est eget nibh volutpat, fermentum aliquet dui mollis.
      Nam vulputate tincidunt fringilla.
      Nullam dignissim ultrices urna non auctor.



\end{description}

\subsubsection{Task list}
\begin{description}
\item[\correctchoice] Lorem ipsum dolor sit amet, consectetur adipiscing elit
\item[\choice] Vestibulum convallis sit amet nisi a tincidunt
\item[\correctchoice] In hac habitasse platea dictumst
\item[\correctchoice] In scelerisque nibh non dolor mollis congue sed et metus
\item[\choice] Praesent sed risus massa
\end{description}

\subsection{Math}
The homomorphism $f$ is injective if and only if its kernel is only the
  singleton set $e_G$, because otherwise $\exists a,b\in
    G$ with $a\neq b$ such
  that $f(a)=f(b)$.


\[ x_{1,2} = \frac{-b \pm \sqrt{b^2 - 4ac}}{2a} \]

\subsection{Abbreviations}
The HTML specification is maintained by the W3C.

We can treat abbreviations as glossaries entries.

\begin{lstlisting}[language=tex]

\documentclass{article}
\usepackage[utf8]{inputenc}
\usepackage[acronym]{glossaries}

\makeglossaries

\newglossaryentry{latex}
{
        name=latex,
        description={Is a mark up language specially suited for
scientific documents}
}

\newglossaryentry{maths}
{
        name=mathematics,
        description={Mathematics is what mathematicians do}
}

\newglossaryentry{formula}
{
        name=formula,
        description={A mathematical expression}
}

\newacronym{gcd}{GCD}{Greatest Common Divisor}

\newacronym{lcm}{LCM}{Least Common Multiple}

\begin{document}

The \Gls{latex} typesetting markup language is specially suitable
for documents that include \gls{maths}. \Glspl{formula} are
rendered properly an easily once one gets used to the commands.

Given a set of numbers, there are elementary methods to compute
its \acrlong{gcd}, which is abbreviated \acrshort{gcd}. This
process is similar to that used for the \acrfull{lcm}.

\clearpage

\printglossary[type=\acronymtype]

\printglossary

\end{document}


        \end{lstlisting}
\subsection{Code}
Some Python Code :

\begin{lstlisting}[language=python]

print("Hello, World!")


        \end{lstlisting}
Code with a title :

\begin{lstlisting}[language=c]
        hello.c
#include &lt;stdio.h&gt;

int main() {
    printf("Hello, World!\n");
    return 0;
}


        \end{lstlisting}
Code with line numbers :

\begin{lstlisting}[language=lisp]





1
2




(defun hello-world ()
  (format t "Hello, World!~%"))






        \end{lstlisting}
Code with highlighting :

\begin{lstlisting}[language=python]

def bubble_sort(items):
    for i in range(len(items)):
        for j in range(len(items) - 1 - i):
            if items[j] &gt; items[j + 1]:
                items[j], items[j + 1] = items[j + 1], items[j]


        \end{lstlisting}
The \lstinline[language=python]|range()| function is
  used to generate a sequence of numbers, also the \lstinline[language=c]|printf()| function is used to print
  a formatted string.

Embedding external files

\begin{lstlisting}[language=c]

#include &lt;stdio.h&gt;

int main(void)
{
    printf("hello, world\n");
    return 0;
}


        \end{lstlisting}
\subsection{Links}
You can go to Wikipédia for more information.

\subsection{Figure}
Here a nice figure :



Figure 1 : Armadillo



\subsection{Table}
This is a table :

<table>
<thead>
<tr>
<th style="text-align: left;">This</th>
<th style="text-align: right;">is</th>
</tr>
</thead>
<tbody>
<tr>
<td style="text-align: left;">a</td>
<td style="text-align: right;">table</td>
</tr>
</tbody>
</table>
\subsection{Admonitions}
<div class="admonition note">
Note

This is a note

</div>
<div class="admonition warning">
Warning

This is a warning with a title

</div>
<details class="question">
<summary>Question</summary>
This is a collapsible question

</details>
<details class="question" open="open">
<summary>Question</summary>
This is a collapsible question expanded

</details>
<div class="admonition info inline end">
Lorem

This is an inline note

</div>
\subsection{Content tabs}

\subsection{Annotations}
One should replace Annotations with footnotes.

Lorem ipsum dolor sit amet, (1) consectetur adipiscing elit.

\begin{enumerate}
\item  I'm an annotation ! I can contain \texttt{code}, \textbf{formatted
      text}, images, ... basically anything that can be expressed in Markdown.
\end{enumerate}

<div class="admonition note annotate">
Phasellus posuere in sem ut cursus (1)

Lorem ipsum dolor sit amet, (2) consectetur adipiscing elit. Nulla et
    euismod nulla. Curabitur feugiat, tortor non consequat finibus, justo
    purus auctor massa, nec semper lorem quam in massa.

</div>
\begin{enumerate}
\item  I'm an annotation !
\item  I'm an annotation as well !
\end{enumerate}

Lorem ipsum dolor sit amet, (1) consectetur adipiscing elit.

\begin{enumerate}
\item  I'm an annotation !
\end{enumerate}

</div>
<div class="tabbed-block">
Phasellus posuere in sem ut cursus (1)

\begin{enumerate}
\item  I'm an annotation as well !
\end{enumerate}

</div>
</div>
</div>

\subsection{Icons}
One method is to parse with MkDocs all the icons, get the svg, include them in place in the latex.



Une icone \includegraphics[width=1em]{7ffa5197e1fbe44b53acd8bded124b5181c77c0acf3e89e37b4377bcf6e81ba6.pdf}

Keyboard keys such as \kbd{Ctrl}+\kbd{Alt}+\kbd{Del} should be supported.

\subsection{SmartSymbols}
We could use ™, ©, ®, ± → ← ↔ ½ ¼ ¾ ⅓ ⅔ ⅛ ⅜ ⅝ ⅞ ⅕ ⅖ ⅗ ⅘ ⅙ ⅚ 1/7 1/9 1/10 from PyMdown Extensions SmartSymbols.

\end{document}