% Taken from https://www.overleaf.com/latex/templates/carauma/pjksmbfyrnkr.pdf
\pagestyle{empty}

% Capa
\begin{center}
    \begin{tikzpicture}[remember picture,overlay]

        % Proporção áurea
        %\pgfmathsetmacro{\goldenheight}{\paperwidth / \goldenratio}
        %\pgfmathsetmacro{\goldenwidth}{\paperwidth}

        % Fundo preto
        \fill[black] (current page.south west) rectangle (current page.north east);

        % Autor
        \node[white, font=\Huge, anchor=north, text width =\linewidth, align = center] (author) at ([yshift=-44pt] current page.north) {\textcolor{customgold}{\scshape\large\authorname}};

        % Título
        \node[white, font=\Large, anchor=north, text width =0.9\linewidth, align = center, below=4pt of author.south] (title) {\scshape\Huge\booktitle};

        % Subtítulo
        \ifx\subtitle\undefined\else\if\relax\detokenize\expandafter{\subtitle}\relax\else
                {\node[white, font=\Large, text width =\linewidth, align = center, anchor=north, below=8pt of title.south] {\textcolor{lightgray}{\scshape\large\subtitle}};}
            \fi\fi

        % Padrão de círculos central
        \begin{scope}[shift={(current page.center)}, scale=1.0871]
            \foreach \x in {-20, -18, -16, -14, -12, -10, -8, -6, -4, -2, 0, 2, 4, 6, 8, 10, 12, 14, 16, 18} {
                    \foreach \y in {-2, 2} {
                            \begin{scope}[shift={(\x + 1, \y)}]
                                \foreach \r in {0.2, 0.4, 0.6, 0.8, 1.0} {
                                        \draw[thick, darkgray] (0,0) circle (\r);
                                    }
                            \end{scope}
                        }
                    \foreach \y in {-3, -1, 1, 3} {
                            \begin{scope}[shift={(\x, \y)}]
                                \foreach \r in {0.2, 0.4} {
                                        \draw[thick, darkgray] (0,0) circle (\r);
                                    }
                            \end{scope}
                        }
                    \foreach \y in {-1, 1} {
                            \begin{scope}[shift={(\x, 0)}]
                                \foreach \r in {0.2, 0.4, 0.6, 0.8, 1.0} {
                                        \draw[thick, customgold] (0,0) circle (\r);
                                        \fill[thick, customgold] (0,0) circle (\r);
                                    }
                            \end{scope}
                        }
                    \foreach \y in {-1, 1} {
                            \begin{scope}[shift={(\x, \y)}]
                                \foreach \r in {0.2, 0.4, 0.6, 0.8, 1.0} {
                                        \fill[black] (0,0) circle (\r);
                                    }
                            \end{scope}
                        }
                    \foreach \y in {-1, 1} {
                            \begin{scope}[shift={(\x, \y)}]
                                \foreach \r in {0.2, 0.4, 0.6, 0.8, 1.0} {
                                        \draw[thick, darkgray] (0,0) circle (\r);
                                    }
                            \end{scope}
                        }
                }
        \end{scope}

        % Editora
        \node[white, font=\Huge, anchor=south] (publisher) at ([yshift=32pt] current page.south) {\textcolor{gray}{\scshape\large\publisher}};

        % Logotipo da editora
        \node[anchor=north, above=4pt of publisher.north] {\includegraphics[width=0.07\textwidth]{frontmatter/logo-white.png}};

        % Outras informações
        \ifx\translatorname\undefined
        \else
            \if\relax\detokenize\expandafter{\translatorname}\relax
                \else{
                    \node[white, font=\Large, anchor=north, above=114pt of publisher.north] (translatedBy) {\textcolor{gray}{\itshape\large{Tradução por}}};

                    \node[white, font=\Large, anchor=south, below=2pt of translatedBy.south] (translatorAuthor) {\textcolor{lightgray}{\itshape\large\translatorname}};
                }
            \fi
        \fi
    \end{tikzpicture}
\end{center}

\cleardoublepage

% Página de título (contra-capa)
\begin{titlepage}
    \centering
    \newgeometry{top=1in,bottom=1in,right=0in,left=0in}
    \thispagestyle{empty}
    ~
    % Autor
    \vspace{24pt}
    {\scshape\large \authorname\par}

    % Título
    \vspace{6pt}
    {\scshape\Huge \booktitle\par}

    % Subtítulo
    \ifx\subtitle\undefined\else\if\relax\detokenize\expandafter{\subtitle}\relax\else
            {\vspace{6pt}
                {\scshape\large \subtitle\par}

                \vspace{\stretch{1.25}}}
        \fi\fi

    % Quem fez a tradução
    \ifx\translatorname\undefined\else\if\relax\detokenize\expandafter{\translatorname}\relax\else
            {{\itshape\large{Tradução por}\par}
                \vspace{6pt}

                {\itshape\Large\translatorname\par}}
        \fi\fi

    % Editora
    \vspace{\stretch{6}}
    {\Huge\scshape\large\publisher\par}
\end{titlepage}
