\documentclass[a4paper,10pt,oneside]{memoir}
\usepackage{fontspec}
\usepackage{imakeidx}
\usepackage[dvipsnames]{xcolor}
\usepackage{makeidx}
\usepackage{glossaries}
\usepackage{lipsum}
\usepackage{tikzducks}
\usepackage{graphicx}
\usepackage{atbegshi}
\usepackage{hyperref}
\usepackage{fmtcount}
\usepackage[backend=biber,style=authoryear]{biblatex}
\usepackage[top=10mm,bottom=10mm]{geometry}

\addbibresource{bibliography.bib}

\newcommand\abspagenumber{\inteval{\ReadonlyShipoutCounter+1}}

\AtBeginDocument{
    \newwrite\pagefile
    \immediate\openout\pagefile=\jobname.pages
}

\makeatletter
\newcounter{volume}
\newcommand{\newvolume}{%
    \stepcounter{volume}%
    \setcounter{figure}{0}
    \setcounter{table}{0}
    \AtBeginShipoutNext{%
        \immediate\write\pagefile{\abspagenumber}%
    }%
    \addtocontents{toc}{\par\centering \bfseries \large Volume \MakeUppercase{\Numberstring{volume}}\par}
}
\makeatother

\AtEndDocument{%
    \immediate\write\pagefile{\abspagenumber}%
    \immediate\closeout\pagefile
}

\makeglossaries
\makeindex[name=vol1idx,title=Index du Volume 1]
\makeindex[name=vol2idx,title=Index du Volume 2]

\newglossaryentry{glossary2}{%
    name={Foo},
    description={Foo is a term that is used to refer to bar.}}
\newglossaryentry{glossary3}{%
    name={Bar},
    description={Bar is a term that is used to refer to foo.}}

\author{Doc Lathrop Brown}

\begin{document}
\raggedbottom
%% Volume 1
\newvolume
\frontmatter
\thispagestyle{empty}
\pagecolor{CadetBlue}
\begin{center}
    \vspace*{5cm}
    {\Huge \bfseries The Book}\\[2cm]
    {\LARGE Volume \MakeUppercase{\Numberstring{volume}}}
\end{center}
\newpage
\nopagecolor

\title{The Book -- Volume \MakeUppercase{\Numberstring{volume}}}
\maketitle
\tableofcontents
\mainmatter
\chapter{Chapter 1}
\section{Introduction Chapter 1}
\index{chapter1}
\lipsum[1]
\chapter{Chapter 2}
\section{Introduction Chapter 2}
\lipsum[2]
\index[vol1idx]{chapter2}
\gls{glossary2}
\parencite{Knuth:1996}
\begin{figure}
    \centering
    \includegraphics{example-image-duck}
    \caption{A figure in the first volume}
\end{figure}
\backmatter
\listoffigures\nobreak\nopagebreak
\listoftables\nobreak\nopagebreak
\printglossary
\printindex[vol1idx]
\printbibliography

%% Volume 2
\clearpage
\newvolume
\thispagestyle{empty}
\pagecolor{MidnightBlue}
\begin{center}
    \vspace*{5cm}
    {\Huge \bfseries The Book}\\[2cm]
    {\LARGE Volume \arabic{volume}}
\end{center}
\newpage
\nopagecolor

\frontmatter
\title{The Book -- Volume \MakeUppercase{\Numberstring{volume}}}
\maketitle
\tableofcontents
\mainmatter
\chapter{Chapter 3}
\section{Introduction Chapter 3}
\index[vol2idx]{chapter3}
\lipsum[3]
\chapter{Chapter 4}
\section{Introduction Chapter 4}
\lipsum[4]
\index{chapter3}
\parencite{latexcompanion}
\gls{glossary3}
\begin{figure}
    \centering
    \includegraphics{example-image-duck}
    \caption{A figure in the first volume}
\end{figure}
\backmatter
\listoffigures\nobreak\nopagebreak
\listoftables\nobreak\nopagebreak
\printglossary
\printindex[vol2idx]
\printbibliography

\end{document}