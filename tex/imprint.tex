
\begingroup
\footnotesize
\setlength{\parindent}{0pt}
\setlength{\parskip}{\baselineskip}

\begin{tabular}{@{} l l}
  \textcopyright{} 2019\:---\:2025 & \author \\
\end{tabular}

Creative Commons \par
\creativecommon{\huge L}

L'œuvre peut être librement utilisée, à la condition de l'attribuer à l'auteur en citant son nom. Le titulaire de droits peut autoriser tous les types d'utilisation ou au contraire restreindre aux utilisations non commerciales (les utilisations commerciales restant soumises à son autorisation). En sommes vous êtes libre de :

\begin{description}
  \item[Partager] copier, distribuer et communiquer le matériel par tous moyens et sous tous formats
  \item[Adapter] remixer, transformer et créer à partir du matériel
\end{description}

Sous les conditions suivantes :

\begin{description}
  \item[Attribution] Vous devez créditer l'Œuvre, intégrer un lien vers la licence et indiquer si des modifications ont été effectuées à l'œuvre. Vous devez indiquer ces informations par tous les moyens raisonnables, sans toutefois suggérer que l'offrant vous soutient ou soutient la façon dont vous avez utilisé son œuvre.
  \item[Partage dans les mêmes conditions] Si vous modifiez, transformez ou adaptez cette œuvre, vous n'avez le droit de distribuer votre création que sous une licence identique ou similaire à celle-ci.
\end{description}

HEIG-VD

Imprimé en Suisse
\vfil
\begin{center}
  10 09 08 07 06 05 04 03 02 01\hspace{2em}19 18 17 16 15 14 13
\end{center}
\begin{center}
  \begin{tabular}{ll}
    Première édition avec Sphinx:                   & 1 Septembre 2019 \\
    Première édition avec Sphinx, avec corrections: & 15 Juillet 2020  \\
    Seconde édition avec MkDocs Material:           & 30 Juillet 2024  \\
  \end{tabular}
\end{center}
\vfil
\endgroup